\documentclass[conference,a4paper]{IEEEtran}

\usepackage[utf8]{inputenc}
\usepackage[english]{babel}
\usepackage[T1]{fontenc}
\usepackage{graphicx,xcolor}
\usepackage{caption}
\usepackage{subcaption}
\usepackage[hidelinks]{hyperref}
\usepackage{amsfonts}

% Benutzen Sie 'hyphenation' fuer eine einfache Silbentrennung
\hyphenation{op-tical net-works semi-conduc-tor}

\begin{document}
% Der Titel Ihrer Seminararbeit
\title{Porting RIOT-OS to the RIoT-Board}

% Version in Englisch 
\author{
\IEEEauthorblockN{Jakob Lennart Dührsen}
\IEEEauthorblockA{Institute of Computer Science\\
Freie Universität Berlin\\
Takustrasse 9, 14195 Berlin, Germany\\
\texttt{lennart.duehrsen@fu-berlin.de}}
\and
\IEEEauthorblockN{Leon Martin George}
\IEEEauthorblockA{Institute of Computer Science\\
Freie Universität Berlin\\
Takustrasse 9, 14195 Berlin, Germany\\
\texttt{leon@georgemail.eu}}
}

% Erzeugen des Titels im Dokument
\maketitle

% Das Abstract sollte nicht mehr als 300 Zeichen umfassen
% \begin{abstract}
% The abstract goes here.
% \end{abstract}

\IEEEpeerreviewmaketitle

In early 2014, Freescale introduced a new board targeting the Internet of 
Things, the \textbf{RIoT-Board}. RIoT is an abbreviation for »Revolutionizing 
the Internet of Things«.\\
Compared to other typical devices used in IoT it is quite powerful, featuring 
an ARM(R) Cortex A9 processor running at 1GHz, 4GB MMC, Gigabit-Ethernet and a 
graphics processor that allows playback of 1080p videos and supports OpenGL ES 
2.0 .\\
\textbf{RIOT OS} on the other hand is an operating system for constrained 
devices, as found frequently in IoT context. It is optimized for low power and 
memory consumption.\\
To enable the operation of RIOT OS on the RIoT-Board, we will port RIOT OS to 
the ARM Cortex A9 architecture as a software project at Freie Universität 
Berlin.\\

Most parts of an OS can be written independently from the underlying hardware, 
which allows for easier portability. Some parts, however, have to be rewritten 
for every new architecture, such as the bootloader and the context switch.\\

As we have the boards in our hands now, the question is where to start. The 
following will give a compendium on what we have already done and are planning 
to do in the near future to let RIOT OS run on the RIoT-Board.\\
Our first step was to get familiar with the board itself. It ships with Android 
pre-installed, so we could boot it immediately and check out some of its 
features. A linux derivative based on Ubuntu is also available, and was testet 
right after Android. Unfortunately, linux can not be booted from the SD card, 
so we had to find out how to flash the board. If you have a Windows computer, 
this is easier than expected. Freescale provides a tool to flash the MMC using 
a mini USB cable. With Linux running, we were able to blink the user LEDs.\\
For further inspection of the RIoT-Board's inner workings, we installed U-Boot, 
a widely used bootloader for embedded systems. U-Boot offers a shell, to which 
you can connect using UART. This set up, we were able to run our own bare-metal 
code.\\

Before coding can start, the hardware-specific parts of RIOT OS have to be 
identified. In the core functionality of RIOT OS, these are the two functions 
\texttt{board\underbar{ }init()} and \texttt{kernel\underbar{ }init()}, which 
are the first to be called during booting and supposed to set up frequencies, 
voltages, stacks and other necessary things to run properly. \\

\newpage

As we are not ARM experts, the first step here is to understand the target 
architecture, which means reading lots of documentation. This work is still in 
progress and probably will be for quite some time.\\
Further things we need to learn include Makefiles and ARM assembly.\\ ~

As soon as we are skilled enough to start working, the plan is to make I/O via 
UART work, to enable comfortable debugging and later access to the shell. After 
that, code to initialize the timers is needed, because the RIOT OS kernel 
depends on these. The lasts steps will then be the \texttt{board\underbar{ 
}init()} and \texttt{kernel\underbar{ }init()}.\\

For that, we have agreed on the following (soft) deadlines:

\begin{itemize}
 \item 31.05. UART and timers working
 \item 31.05. \texttt{board\underbar{ }init()} working
 \item 31.06. \texttt{kernel\underbar{ }init()} working
 \item 07.07. more realistic deadline for the above
\end{itemize}

As of today, we are not able to tell how the work will be split, because the 
exact approach for each milestone is not clear yet. So far, our attitude was to 
identify the next small task or problem, respectively, start reading and 
hacking and whoever finds a solution first explains it to his teammate and we 
start again. This has worked pretty good, so we decided to keep it that way 
until we know what we're doing by splitting work in a planned way.





% Hiermit legen Sie den Style fuer das Literaturverzeichnis fest
% \bibliographystyle{unsrt}
% Hier wird die BibTeX Datei (references.bib) eingebunden
% \bibliography{references}

\end{document}
