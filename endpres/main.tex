\documentclass[9pt]{beamer}

\usepackage[utf8]{inputenc}
\usepackage{mathtools}
\usepackage{hyperref}
\usepackage{color}
\definecolor{light-gray}{gray}{0.90}

 % Template for talks using the Corporate Design of the Freie Universitaet
%   Berlin, created following the guidelines on www.fu-berlin.de/cd by
%   Tobias G. Pfeiffer, <tobias.pfeiffer@math.fu-berlin.de>
% This file can be redistributed and/or modified in any way you like.
%   If you feel you have done significant improvements to this template,
%   please consider providing your modified version to
%   https://www.mi.fu-berlin.de/w/Mi/BeamerTemplateCorporateDesign

\usepackage{listings}

%%% FU logo
% small version for upper right corner of normal pages
\pgfdeclareimage[height=0.9cm]{university-logo}{FULogo_RGB}
\logo{\pgfuseimage{university-logo}}
% large version for upper right corner of title page
\pgfdeclareimage[height=1.085cm]{big-university-logo}{FULogo_RGB}
%%% end FU logo

% NOTE: 1cm = 0.393 in = 28.346 pt;    1 pt = 1/72 in = 0.0352 cm
\setbeamersize{text margin right=3.5mm, text margin left=7.5mm}  % text margin

% colors to be used
\definecolor{text-grey}{rgb}{0.45, 0.45, 0.45} % grey text on white background
\definecolor{bg-grey}{rgb}{0.66, 0.65, 0.60} % grey background (for white text)
\definecolor{fu-blue}{RGB}{0, 51, 102} % blue text
\definecolor{fu-green}{RGB}{153, 204, 0} % green text
\definecolor{fu-red}{RGB}{204, 0, 0} % red text (used by \alert)

% switch off the sidebars
% TODO: loading \useoutertheme{sidebar} (which is maybe wanted) also inserts
%   a sidebar on title page (unwanted), also indents the page title (unwanted?),
%   and duplicates the navigation symbols (unwanted)
\setbeamersize{sidebar width left=0cm, sidebar width right=0mm}
\setbeamertemplate{sidebar right}{}
\setbeamertemplate{sidebar left}{}
%    XOR
% \useoutertheme{sidebar}

% frame title
% is truncated before logo and splits on two lines
% if neccessary (or manually using \\)
\setbeamertemplate{frametitle}{%
    \vskip-30pt \color{text-grey}\large%
    \begin{minipage}[b][23pt]{80.5mm}%
    \flushleft\insertframetitle%
    \end{minipage}%
}

%%% title page
% TODO: get rid of the navigation symbols on the title page.
%   actually, \frame[plain] *should* remove them...
\setbeamertemplate{title page}{
% upper right: FU logo
\vskip2pt\hfill\pgfuseimage{big-university-logo} \\
\vskip6pt\hskip3pt
% title image of the presentation
\begin{minipage}{11.6cm}
\hspace{-1mm}\inserttitlegraphic
\end{minipage}

% set the title and the author
\vskip14pt
\parbox[top][3cm][c]{14cm}{\color{text-grey}\inserttitle \\ \small \insertsubtitle}
\vskip11pt
\parbox[top][1.35cm][c]{11cm}{\small \insertauthor \\ \insertinstitute \\[3mm] \insertdate}
}
%%% end title page

%%% colors
\usecolortheme{lily}
\setbeamercolor*{normal text}{fg=black,bg=white}
\setbeamercolor*{alerted text}{fg=fu-red}
\setbeamercolor*{example text}{fg=fu-green}
\setbeamercolor*{structure}{fg=fu-blue}

\setbeamercolor*{block title}{fg=white,bg=black!50}
\setbeamercolor*{block title alerted}{fg=white,bg=black!50}
\setbeamercolor*{block title example}{fg=white,bg=black!50}

\setbeamercolor*{block body}{bg=black!10}
\setbeamercolor*{block body alerted}{bg=black!10}
\setbeamercolor*{block body example}{bg=black!10}

\setbeamercolor{bibliography entry author}{fg=fu-blue}
% TODO: this doesn't work at all:
\setbeamercolor{bibliography entry journal}{fg=text-grey}

\setbeamercolor{item}{fg=fu-blue}
\setbeamercolor{navigation symbols}{fg=text-grey,bg=bg-grey}
%%% end colors

%%% headline
\setbeamertemplate{headline}{
\vskip4pt\hfill\insertlogo\hspace{3.5mm} % logo on the right

\vskip6pt\color{fu-blue}\rule{\textwidth}{0.4pt} % horizontal line
}
%%% end headline

%%% footline
\newcommand{\footlinetext}{\insertshortinstitute, \insertshorttitle, \insertshortdate}
\setbeamertemplate{footline}{
\vskip5pt\color{fu-blue}\rule{\textwidth}{0.4pt}\\ % horizontal line
\vskip2pt
\makebox[123mm]{\hspace{7.5mm}
\color{fu-blue}\footlinetext
\hfill \raisebox{-1pt}{\usebeamertemplate***{navigation symbols}}
\hfill \insertframenumber}
\vskip4pt
}
%%% end footline

%%% settings for listings package
\lstset{extendedchars=true, showstringspaces=false, basicstyle=\footnotesize\sffamily, tabsize=2, breaklines=true, breakindent=10pt, frame=l, columns=fullflexible}
\lstset{language=Java} % this sets the syntax highlighting
\lstset{mathescape=true} % this switches on $...$ substitution in code
% enables UTF-8 in source code:
\lstset{literate={ä}{{\"a}}1 {ö}{{\"o}}1 {ü}{{\"u}}1 {Ä}{{\"A}}1 {Ö}{{\"O}}1 {Ü}{{\"U}}1 {ß}{\ss}1}
%%% end listings

\newcommand{\biframe}[2]{
\begin{frame}{#1} 
\begin{itemize}
#2
\end{itemize}
\end{frame}
}


\title{Bringing RIoT-OS to the RIoTboard}
 % \subtitle{Zweite Zw.-Pres.}

\author{Lennart Dührsen and Leon Martin George}

 \institute[FU Berlin]{Freie Universität Berlin}

 \date{Softwareproject - Telematics, 2014}

 \subject{\fontsize{15cm}{1em}Computer Science}

 \renewcommand{\footlinetext}{\insertshortinstitute, \insertshorttitle, \insertshortdate}
 \newcommand{\tableofsubs}{\tableofcontents[currentsection,sectionstyle=show/show,subsectionstyle=show/shaded/hide]}

\begin{document}

 \begin{frame}[plain]
  \titlepage
% Were porting to the RIoTboard. You probably know us by now...
 \end{frame}

\section{Motivation}

\begin{frame}{Context}
 \begin{itemize}
% also commonly known as: META-stuff-slide
\pause
 \item what do we want?
\pause
 \begin{itemize}
 \item RIoT-OS running on the RIoTboard
% because, why not?
\pause
 \item have fun coding
% who doesn't?
\pause
 \item fancy hardware
% thank you (for the hardware)
\pause
 \item credit points
% why wouldn't we?
 \end{itemize}
\pause
 \item what did we expect to achieve?
\pause
 \begin{itemize}
 \item get the hardware for free
% because that's the only real reason to join a telematics-project
\pause
 \item basic support of the RIoTboard for RIoT-OS
% just because these names go really well together..
\pause
 \item be motivated enough to continue working on the port after the software-project
% seems like it's far more interesting than "deytabeysis" + plus we can keep the board for as long as we are working on it
 \end{itemize}
 \end{itemize}
\end{frame}

\section{Task-Division}

\begin{frame}{Task-Division}
% aka "boring slide"
\pause
 \begin{itemize}
 \item sub-goals:
% we had help planning the steps before the project even started
\pause
 \begin{itemize}
 \item gather relevant documents
% if only we had read them also...
\pause
 \item find out how on the board "works"
% we asked for the hardware early and having the board and a linux-port really helps
\pause
 \item build a basic application that runs on the board (LED-blinking)
% get something to run
\pause
 \item build this application from within RIoT-OS (run our program from the \texttt{board\_init})
% build "something" the RIoT-style
\pause
 \item enable interrupts
% interrupts are enabled
\pause
 \item UART for STDIO
% stdio is somehow nice to have
\pause
 \item implement timer-interface
% there are two....
\pause
 \item wiki pages
% have wiki pages in our github-fork
\pause
 \end{itemize}
 \item milestone arrangement
% these points really don't make sense in any other order
\pause
 \item milestones have dates assigned
% the team urged us not to try to do too much and we were very conservative with the dates
\pause
 \item milestones are coarse
% and we expected to be able to split them ad-hoc
 \end{itemize}
\end{frame}

\begin{frame}{Milestones}
% these are the milestones
 \begin{itemize}
 \item get familiar with the board
 	\begin{itemize}
 	\item boot it, read manuals and documentation
 	\item try features with existing OS that supports it
 	\item understand target architecture
 	\item flash it
 	\item cross-compile
 	\item be able to actually run bare-metal code
 	\end{itemize}
 \item find out what needs to be done for a port
 	\begin{itemize}
 	\item identify re-useable code
% |let them read a bit|
% identify re-useable code: we tried to use the u-boot-port and ... panic mode! \\
 	\item learn about interfaces in RIoT
% learn about interfaces in RIoT: skipped this and ran straight to adjusting the code of the SDK
 	\end{itemize}
 \item port it
 	\begin{itemize}
 	\item patch SDK for use in RIoT
 	\item successfully build
 	\item debug
 	\end{itemize}
 \end{itemize}
\end{frame}


\section{Outcome}

\begin{frame}{Outcome}
\pause
 \begin{itemize}
  \item all goals reached
% but..
\pause
 	\item spaghetti
% we have working code that is in an "interesting" state
 \end{itemize}
\end{frame}

\begin{frame}{Problems - documentation}
% or: did everything work as expected?
% short answer - as with every project: no (or nöööö)
% while gathering documents we were expecting something like a straight-forward
% memory-map like we were used to from the micro-processor-lab
\pause
 \begin{itemize}
 \item expectation: to turn on the LED write a bit to \$beef1337:3
% warning: the ugliness of these slides depicts the ugliness of having to find out these things the hard way
\pause
 \item reality:
 	\begin{itemize}
 \item i.MX6-reference-manual
 		\begin{itemize}
 \item the status of a GPIO-pin is determined by a bit in a register that can be anywhere - based on the configuration of the muxer
 \item names of channels in the muxer are from the same namespace as the functions mapped onto them
 		\end{itemize}
 \item RIoTboard-schematics
 		\begin{itemize}
 \item one LED on the RIoTboard is connected to a function "EIM\_DATAwx" \textcolor{light-gray}{which you can then map GPIOyz on}
 		\end{itemize}
 	\end{itemize}
 \end{itemize}
% this is the same for any other component: there are eight DIP-switches on the board:
% go to the schematics. switches are numbered from 0 to 7 and have functions assigned (FUSEs).
% FUSEs are described in the reference manual.
% But the physical switches are numbered from 1 to 8.
% So => back to the quick-start-guide where there are sample-configurations for flashing
% UART it says nowhere which of the three PINs is which.
\end{frame}

\begin{frame}{Problems - reference code}
 \begin{itemize}
\pause
 \item embest-tech doesn't supply a muxer-configuration-file that can be used with the SDK
% which is kinda sad. They have targeted mainly android development and a bit of linux
\pause
 \item maybe looking at how they did it for their u-boot- and linux-ports helps?
% is spaghetti. we don't dare think about how our code would look like if be based it on spaghetti.
\pause
 \item reconsideration: the i.MX6-platform-SDK has macros to abstract to and from the muxer config
% which you can only use if you have valid configuration file which in turn can only be created and generated code from with a windows-only-tool
 \end{itemize}
\end{frame}

\begin{frame}{Problems - different abstractions}
 \begin{itemize}
\pause
 \item The platform-SDK differentiates between code concerning
% then there is a problem of different abstractions in the SDK and in RIoT
\pause
 	\begin{itemize}
 	\item the i.MX6-architecture
% CPU, interrupt-controller
\pause
 	\item peripherals
% driver for UART and timers (which in RIoT belong to the CPU) or ethernet and USB
\pause
 	\item board-specific headers and iomux-configuration
% iomux-configuration, register definitions (depending on the kind of i.MX6 - SL - SDL - D)
\pause
 	\end{itemize}
 \item RIoT has its own abstraction for each of those (dividing the sub-topics differently)
% We should we include the SDK ?
 \end{itemize}
\end{frame}

\begin{frame}{Demo}
DEMO (of \texttt{printf}s and flashing LEDs)
\end{frame}

\begin{frame}{PRs}
% At the time of writing we have one failed, a successful and another pending pull request:
 \begin{itemize}
 \item \#1355 was closed in favour of \#1359 \textcolor{light-gray}{Leon had trouble keeping the git-log tidy}
% This would have allowed using LD for linking by changing the global Makefile and the
% Makefile of any board not using LD (so far: all except the RIoTboard).
 \item \#1359 was merged - with the help of staff members and RIoT-maintainers.
% This basically does the same as \#1355 but leaves other Makefiles untouched and
% assuming GCC to be used by default. Boards using LD have to explicitely supply a
% variable.
 \item \#1411 is still pending and we do not know whether it will be merged.
% It's purpose is to bring the software-project to a conclusion and add support for
% the riotboard and the i.MX6-SDK.
% This contains hundreds of file from the SDK that have been altered to just work.
% (obviously not fit for rolling out)
 \end{itemize}
\end{frame}

\section{Future}

\begin{frame}{Perspectives}
 \begin{itemize}
 \item still motivated
 \item perhaps scrap the existing code and restart
% restart in two weeks of time
 \item or: clean the existing code
 \end{itemize}
\end{frame}

\begin{frame}{Questions?}

\end{frame}


\end{document}
