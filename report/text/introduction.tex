\section{Introduction}
In early 2014, Freescale Semiconductors \cite{freescale-web} introduced a new 
open-source single-board computer targeting the Internet of Things, the 
RIoT-Board \cite{riot-web}. »RIoT« is an abbreviation for 
»Revolutionizing the Internet of Things«, which the board targets.
Compared to other boards typically used in IoT context, such as the TI MSP430 
\cite{msp430} or the XMC 2Go \cite{xmc}, it is quite powerful and thus suitable 
for applications requiring high levels of processing power.

RIOT OS \cite{riot-os} on the other hand, using the same abbreviation in its 
name, is an open-source operating system developed by Freie Universität Berlin 
\cite{fu-web}, INRIA \cite{inria-web} and Hamburg University of Applied 
Sciences \cite{haw-web}. It was designed for constrained devices often used in 
IoT, providing partial POSIX compliance and realtime capabilities.

The RIoTboard provides a lot more memory and computation power needed for RIOT 
OS to run properly, but this is also what makes it interesting: With the 
RIoTboard underneath, a whole new range of applications could run on RIOT OS, 
e.g. central data processing in a sensor network or even graphical applications.

This is why we decided to port RIOT OS to the RIoTboard as part of the 
lab-course »Telematics-Project« \cite{course-web} at Freie Universität Berlin 
during the summer term 2014.

The following sections will provide more information how to port RIOT OS to new 
platforms, planned milestones and our further proceedings.

All the files we used while porting that are not used for building, such as 
documents, source code for working software, instructions and overviews, were 
gathered in an additional \texttt{git} repository \cite{git-report}.
