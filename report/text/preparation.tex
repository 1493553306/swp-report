\section{Preparing For The Port}

In order to complete the project successfully, we agreed on these goals with the
teaching-staff:

\begin{itemize}
\item Create wiki pages on working with the hardware
\item Compile RIOT-OS for the RIoTboard
\item Implement interfaces for timers, UART and interrupts
\end{itemize}

As we had no idea how hard it was going to be, we were very generous with the 
curfews
for the different steps.

\subsection{Hardware and Information}

The manufacturer supplies two relevant documents: A 
Quick-User-Guide/User-Manual/
Technical-Data-Sheet and the board schematics.
The board-manual consists mainly of a list of interfaces with picture of where 
to find them on the board. If you have physical access to the board, the only 
thing useful you will learn from it is which mini-USB-port is for OpenSDA 
and which is for the serial-connection.
At the end there is a step-by-step instruction on "flashing" linux or android 
onto the board, by using a Windows-only tool, arranging the eight DIP-switches 
on the board in three different ways.
The schematics file yields a bit more information, such as which pin is 
connected to what.
Other pieces of information, like which pin of UART2 is GND/TXD/RXD, is missing.
You can find that information in the repository mentioned in the introduction
(\textit{$briefings/connect\_debug$}).
In order to understand the way the board works the i.MX6SDL processor reference 
manual is the most helpful. All the interfaces and the processor, including the 
boot-process, are described in there.

\subsection{Software Running on the RIoTboard}
On the boards website the manufacturer - embest-tech - offers binary-images and
instructions for running android or ubuntu, of which android is installed by 
default.
Both operating systems rely on u-boot as the boot-loader. The source-code of 
those ports is available in moderately-hard-to-find repositories 
\cite{git-uboot} on the internet and the code itself may somehow be re-useable.

Due to their structure, the linux-code and the u-boot-fork were useful in 
different ways:
u-boot requires the seperation of syscalls, the "flash-header" - which helped 
understand the way the i.MX6 boots - and linker script.
So with u-boot-imx embest-tech provides three files for those and one giant 
source-file for all the rest.
The linux-port seems better structured but due our lack of knowledge about the
kernel-nuild-process it was hard to use any parts of the source.
Also, as far as initialisation goes, linux seems to repeat some of the steps 
that have already been done by u-boot and does some differently.

The focus on developers is somehow restricted to those two operating systems. 
There is no official support apart from discussions on a channel of an 
IoT/embedded-focused developer forum \cite{forum}.

\subsection{The Attempt of using u-boot}
We really lost a lot of time on our attempt at basing the port on u-boot-code.
During most of that time we did not produce anything that was really relevant 
because we had problems figuring out, how u-boot works during the build-process.
After three weeks the instructors gave us the advice to look for an alternative 
and that even though we had not been able to run anything with it, the SDK 
looked more promising.
