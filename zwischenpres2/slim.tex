\documentclass[9pt]{beamer}

\usepackage[utf8]{inputenc}
\usepackage{mathtools}
\usepackage{hyperref}

 % Template for talks using the Corporate Design of the Freie Universitaet
%   Berlin, created following the guidelines on www.fu-berlin.de/cd by
%   Tobias G. Pfeiffer, <tobias.pfeiffer@math.fu-berlin.de>
% This file can be redistributed and/or modified in any way you like.
%   If you feel you have done significant improvements to this template,
%   please consider providing your modified version to
%   https://www.mi.fu-berlin.de/w/Mi/BeamerTemplateCorporateDesign

\usepackage{listings}

%%% FU logo
% small version for upper right corner of normal pages
\pgfdeclareimage[height=0.9cm]{university-logo}{FULogo_RGB}
\logo{\pgfuseimage{university-logo}}
% large version for upper right corner of title page
\pgfdeclareimage[height=1.085cm]{big-university-logo}{FULogo_RGB}
%%% end FU logo

% NOTE: 1cm = 0.393 in = 28.346 pt;    1 pt = 1/72 in = 0.0352 cm
\setbeamersize{text margin right=3.5mm, text margin left=7.5mm}  % text margin

% colors to be used
\definecolor{text-grey}{rgb}{0.45, 0.45, 0.45} % grey text on white background
\definecolor{bg-grey}{rgb}{0.66, 0.65, 0.60} % grey background (for white text)
\definecolor{fu-blue}{RGB}{0, 51, 102} % blue text
\definecolor{fu-green}{RGB}{153, 204, 0} % green text
\definecolor{fu-red}{RGB}{204, 0, 0} % red text (used by \alert)

% switch off the sidebars
% TODO: loading \useoutertheme{sidebar} (which is maybe wanted) also inserts
%   a sidebar on title page (unwanted), also indents the page title (unwanted?),
%   and duplicates the navigation symbols (unwanted)
\setbeamersize{sidebar width left=0cm, sidebar width right=0mm}
\setbeamertemplate{sidebar right}{}
\setbeamertemplate{sidebar left}{}
%    XOR
% \useoutertheme{sidebar}

% frame title
% is truncated before logo and splits on two lines
% if neccessary (or manually using \\)
\setbeamertemplate{frametitle}{%
    \vskip-30pt \color{text-grey}\large%
    \begin{minipage}[b][23pt]{80.5mm}%
    \flushleft\insertframetitle%
    \end{minipage}%
}

%%% title page
% TODO: get rid of the navigation symbols on the title page.
%   actually, \frame[plain] *should* remove them...
\setbeamertemplate{title page}{
% upper right: FU logo
\vskip2pt\hfill\pgfuseimage{big-university-logo} \\
\vskip6pt\hskip3pt
% title image of the presentation
\begin{minipage}{11.6cm}
\hspace{-1mm}\inserttitlegraphic
\end{minipage}

% set the title and the author
\vskip14pt
\parbox[top][3cm][c]{14cm}{\color{text-grey}\inserttitle \\ \small \insertsubtitle}
\vskip11pt
\parbox[top][1.35cm][c]{11cm}{\small \insertauthor \\ \insertinstitute \\[3mm] \insertdate}
}
%%% end title page

%%% colors
\usecolortheme{lily}
\setbeamercolor*{normal text}{fg=black,bg=white}
\setbeamercolor*{alerted text}{fg=fu-red}
\setbeamercolor*{example text}{fg=fu-green}
\setbeamercolor*{structure}{fg=fu-blue}

\setbeamercolor*{block title}{fg=white,bg=black!50}
\setbeamercolor*{block title alerted}{fg=white,bg=black!50}
\setbeamercolor*{block title example}{fg=white,bg=black!50}

\setbeamercolor*{block body}{bg=black!10}
\setbeamercolor*{block body alerted}{bg=black!10}
\setbeamercolor*{block body example}{bg=black!10}

\setbeamercolor{bibliography entry author}{fg=fu-blue}
% TODO: this doesn't work at all:
\setbeamercolor{bibliography entry journal}{fg=text-grey}

\setbeamercolor{item}{fg=fu-blue}
\setbeamercolor{navigation symbols}{fg=text-grey,bg=bg-grey}
%%% end colors

%%% headline
\setbeamertemplate{headline}{
\vskip4pt\hfill\insertlogo\hspace{3.5mm} % logo on the right

\vskip6pt\color{fu-blue}\rule{\textwidth}{0.4pt} % horizontal line
}
%%% end headline

%%% footline
\newcommand{\footlinetext}{\insertshortinstitute, \insertshorttitle, \insertshortdate}
\setbeamertemplate{footline}{
\vskip5pt\color{fu-blue}\rule{\textwidth}{0.4pt}\\ % horizontal line
\vskip2pt
\makebox[123mm]{\hspace{7.5mm}
\color{fu-blue}\footlinetext
\hfill \raisebox{-1pt}{\usebeamertemplate***{navigation symbols}}
\hfill \insertframenumber}
\vskip4pt
}
%%% end footline

%%% settings for listings package
\lstset{extendedchars=true, showstringspaces=false, basicstyle=\footnotesize\sffamily, tabsize=2, breaklines=true, breakindent=10pt, frame=l, columns=fullflexible}
\lstset{language=Java} % this sets the syntax highlighting
\lstset{mathescape=true} % this switches on $...$ substitution in code
% enables UTF-8 in source code:
\lstset{literate={ä}{{\"a}}1 {ö}{{\"o}}1 {ü}{{\"u}}1 {Ä}{{\"A}}1 {Ö}{{\"O}}1 {Ü}{{\"U}}1 {ß}{\ss}1}
%%% end listings

\newcommand{\biframe}[2]{
\begin{frame}{#1} 
\begin{itemize}
#2
\end{itemize}
\end{frame}
}


\title{Bringing RIoT-OS to the RIoTboard}
 % \subtitle{Zweite Zw.-Pres.}

\author{Leon Martin George}

 \institute[FU Berlin]{Freie Universität Berlin}

 \date{Softwareproject - Telematics, 2014}

 \subject{\fontsize{15cm}{1em}Computer Science}

 \renewcommand{\footlinetext}{\insertshortinstitute, \insertshorttitle, \insertshortdate}
 \newcommand{\tableofsubs}{\tableofcontents[currentsection,sectionstyle=show/show,subsectionstyle=show/shaded/hide]}

\begin{document}

 \begin{frame}[plain]
  \titlepage
% \titlepage
 \end{frame}

\section{Motivation}

\begin{frame}{The Board}
 \begin{itemize}
 \item Designed for developing
% comes with instructions for android and linux. android preinstalled
 \item Cortex-A9-based
% 1 GHz, single-core\\
 \item Freescale i.MX6-architecture
% SOC for mm-applications\\
 \item Co-processing power
% chips for decoding x264, 2d- and 3d-graphics\\
 \item Many different interfaces
% USB, eth, sound, HDMI, Display-Port, dev-stuff like JTAG, UART
% Can boot from internal memory, or SD-card
 \end{itemize}
\end{frame}

\begin{frame}{Why Do We Want This?}
% RIOT-OS is not known for it's need to run on phones.
% With i.MX6-support RIoT-OS could be used for 
 \begin{itemize}
 \item Automotive 
% automotive,\\
 \item Industrial 
% industrial,\\
 \item Handheld consoles
% handheld consoles,
 \item Easy developement
% easy developement->This often is underestimated. Most microprocessors are
% painful to target for developing. The RIoTboard is aimed at developers: You
% can have different programs on the board itself or on several SD-cards.
% Flashing can be done to either targets via USB or directly to an SD-Card
% without having to destroy the data on the partitions (after some preparation).
% The fully-developed program can probably be used for any other i.MX6-board
% without too many modifications, if any (maybe IOMux brrrrrrrrrrr).
 \end{itemize}
\end{frame}

\section{Our original idea of how to do shit}
\begin{frame}{Initial Work-Model}
 \begin{itemize}
 \item Assess the situation
% This was easy. Many documents.
 \item Try running anything on the board
% We have to documents from the manufacturer: One is a schematic and the other
% is specification for the hardware with instructions on flashing either the SD
% -card or the internal eMMC (micro-SD they forgot somehow) with ubuntu or android.
% Glad there were instructions for this.
 \item Run our own code
% After having realised that we wouldn't be able to flash the ROM, we tried
% going through u-boot and loaded just a main() onto the board.
 \item Get a framework to run
% We wanted to use u-boot but the IOMux-configuration would then have been
% fixed and it is usually a bad idea to change that (what RIoT would have
% ultimately done). Plus, the teaching staff convinced us that the i.MX6-SDK
% is much nicer. Somehow, we have got a working framework.
 \item Split to work on different components individually
% Even though we could, in theory, start getting specific hardware to work
% with RIoT-OS, there is a significant problem.
 \end{itemize}
\end{frame}

\section{Achievements}

\begin{frame}{The Original Plan:}
% from the orinal slides:
 \begin{itemize}
 \item UART I/O for debugging and shell communication
% uart. erm. problems. everything set up, but no cigar\\
 \item Timer(s) so the kernel can run
% timers are running by default, but not yet usable by RIoT (or in fact: the
% UART)\\
 \item Interrupts
% interrupts are enable, but no implementation for interface to RIoT\\
 \item Set up a stack
% set-up stack, done (works nicely due to the SDK)\\
 \item Build it successfully (probably the hardest part :-) )
% This worked after about a month
 \end{itemize}
\end{frame}

\begin{frame}{Unfinished Work}
% The slide should rather be called "The One Big Problem" because there is this
% one thing that is really bugging us:
 \begin{itemize}
 \item The UART
% We tried the default configuration from the SDK, diy-UART-init, inserted
% adapted u-boot-code, tried configs for other boards and made some ourselves
% with the IOMux-config-tool from the SDK which generates headers with register
% -definitions and macro-implementations.\\
 \item Timers
% Then there is timers
 \item Interrupts
% and interrupts\\
 \end{itemize}
% but with the only working debugging facilities being to LEDs, this is a
% rather terrifying task
\end{frame}

\begin{frame}{Interference}
 \begin{itemize}
 \item Going from u-boot to SDK
% This was initially a drawback because it took us significantly longer to
% integrate the SDK than it would have with u-boot(mainly because the board-
% specific code from u-boot is squeezed into just 4 source-files.
% But after the make-process of the SDK was integrated into RIoT-OS it was
% really easy to adjust the start-process. So by then we were early.
 \item The UART initialisation process
% Nuff said on that.
 \end{itemize}
\end{frame}

\section{Future Tense}

\begin{frame}{Plans}
 \begin{itemize}
 \item Try altering IOMux-configurations for other boards
% Other IOMux-confs might offer insight on what is going wrong
 \item JTAG-debugging
% Avoid the UART-related problems to allow getting to work on other components
 \end{itemize}
\end{frame}

\end{document}
